% Modified from the following template
% Commutative diagram with edges passing under/over
% Jan 7, 2009, Stefan Kottwitz
% http://texblog.net

\documentclass[a4paper,landscape]{scrartcl}
%%%<
\usepackage{verbatim}
\usepackage[active,tightpage]{preview}
%\PreviewEnvironment{tikzpicture}
\setlength\PreviewBorder{5pt}%
%%%>
\usepackage{tikz}
\usetikzlibrary{matrix}

\definecolor{green}{rgb}{0.05,0.54,0.25}
\newcommand{\edgebayes}{edge[blue, ultra thick]}
\newcommand{\edgewhite}{edge[-,line width=6pt,draw=white]}
\newcommand{\edgebag}{edge[green, ultra thick]}
\newcommand{\edgeboost}{edge[red, ultra thick]}

%\newcommand[1]{\textstyle}{\textsf{#1}}
\def\mytextstyle{\Large\textsf}

\begin{document}
\begin{preview}
\begin{tikzpicture}%[every node/.style={anchor=base,text depth=.5ex,text height=2ex,text width=1em}]
      \matrix (m) [matrix of nodes, row sep=3em,
    column sep=1em]{
    \mytextstyle{Decision trees}& & \mytextstyle{Bayesian decision trees} &  \\
    & \mytextstyle{Random forests} &  &  \mytextstyle{?} \\
   \mytextstyle{Boosted  trees} &  & \mytextstyle{BART} &  \\
     & \mytextstyle{BagBoo} &  & \mytextstyle{?} \\};
  \path[-stealth]
    (m-1-1) \edgebayes (m-1-3) \edgebag (m-2-2) \edgeboost (m-3-1)
    (m-1-3) \edgeboost (m-3-3) \edgebag (m-2-4)
        (m-2-2) \edgewhite (m-2-4) \edgebayes (m-2-4)
    (m-3-1) \edgebayes (m-3-3)    \edgebag (m-4-2)
              (m-2-2) \edgewhite (m-4-2) \edgeboost (m-4-2)
 (m-4-2) \edgebayes (m-4-4)
    (m-3-3) \edgebag (m-4-4)
    (m-2-4) \edgeboost (m-4-4)
            ;
\end{tikzpicture}
\end{preview}
\end{document}